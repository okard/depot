\documentclass{article}
\usepackage[a4paper]{geometry}
\usepackage[utf8]{inputenc}
\usepackage[T1]{fontenc}
\usepackage[ngerman]{babel}
\usepackage{amsmath}
\usepackage{MnSymbol}
\usepackage{mathptmx} %times font
\usepackage{graphicx}
\usepackage{subfig}
\usepackage{color}
\usepackage{hyperref}
\usepackage{wrapfig}
 
\title{Brettspiel}
\author{Okard}
\date{\today}

\definecolor{silver}{rgb}{0.937,0.90,0.937} %239 232 239
\linespread {1.25}

\begin{document}

\maketitle
\tableofcontents
\newpage

%%%%%%%%%%%%%%%%%%%%%%%%%%%%%%%%%%%%%%%%%%%%%%%%%%%%%%%%%%%%
%% Einführung
%%%%%%%%%%%%%%%%%%%%%%%%%%%%%%%%%%%%%%%%%%%%%%%%%%%%%%%%%%%%
\section{Einführung}
	Es handelt sich hier um ein strategisches Brettspiel wie Schach oder Shôgi.
	Gespielt wird auf einem kariertem Spielbrett mit Spielsteinen.
	Im Gegensatz zu den genannten Spielen herrscht eine größere Individualität.
	Jeder Spieler kann seine 'Armee' nach seinen Bedürfnissen zusammenstellen, die
	Anzahl und Art der Spielsteine kann also variieren und ist nicht fest vorgegeben.
	Zusätzlich wird als Zufallselement ein Würfel verwendet, was ebenso mit einkalkuliert 
	werden muss.
	Die Regeln sind so konstruiert das man sie leicht verstehen kann. Um das Spiel meistern
	zu können muss allerdings jedes kleinste Detail bedacht werden.
	
	%Schach -> Krieg auf dem Spielbrett
	%NAMEN
		% Pugna? lat. Kampf/Schlacht
	%Entstehung des Spiels
	
	%Inspiriert durch Schach, Shogi
	
	%Taktik, Strategie mit ein wenig Zufall
	%Individualität durch Spielstein wahl


%%%%%%%%%%%%%%%%%%%%%%%%%%%%%%%%%%%%%%%%%%%%%%%%%%%%%%%%%%%%
%% Spielsteine & Spielzüge
%%%%%%%%%%%%%%%%%%%%%%%%%%%%%%%%%%%%%%%%%%%%%%%%%%%%%%%%%%%%
\section{Spielsteine \& Spielzüge}

\subsection{Der Spielstein}
		
	\begin{wrapfigure}{l}{5cm}
	\centering
	\includegraphics[scale=2]{pdf/token.pdf}
	\caption{Spielstein Vorder- und Rückseite}
	\end{wrapfigure}
	
	Der Spielstein ist dem japanischem Sh\=ogi Spielstein nachempfunden. Beide Seiten eines Spielsteins können belegt sein.
	Die Spitze zeigt die Spielrichtung und somit die Zugehörigkeit zum Spieler an.
	Die äußeren Kreise an den Kreuzenden geben an in welche Richtung der Spielstein sich bewegen und schlagen kann.
	Unten links ist in jedem Spielstein dessen Stärke und auf der Frontseite unten rechts die Stärke der Rückseite angegeben.
	Die Stärke des Spielsteins bestimmt wie viel Aufwand nötig ist um mit diesem Spielstein einen anderen zu schlagen.
	\\

	\begin{flushleft}
	Die Spielstärke definiert sich wie folgt:	
	
	\begin{tabular}{lcl}
	\hline
	Symbol & Wertigkeit & Name \\ \hline 
	$\blacksquare$ & 1 & Bauer \\ 
	$\blacksquare\blacksquare$ & 2 & Ritter \\ 
	$\blacksquare\blacksquare\blacksquare$ & 3 & General \\ 
	\end{tabular}
	\end{flushleft}
	
	
	
	\begin{wrapfigure}{r}{6.2cm}
  	\includegraphics[scale=1]{pdf/token_on_board.pdf}
  	\caption{Spielstein und Brett}
	\end{wrapfigure}
	
	Die maximal 8 äußeren Kreise geben an in welche Richtungen der Spielstein ziehen und 
	schlagen darf, sie beziehen sich auf die Felder die um den Spielstein liegen.
	Wenn ein Spielstein in einer Richtung schlagen darf, so darf er sich auch in 
	diese Richtung bewegen ohne eine andere Figur schlagen zu müssen. Falls für eine
	Richtung kein Kreis vorhanden ist bedeutet dies das der Spielstein in diese Richtung
	weder schlagen noch ziehen darf. Durch eine Beförderung des Spielsteins, also das umdrehen
	auf die untere Seite kann sich das Zugverhalten eines Spielsteins jedoch ändern.
	
	
	\vspace{0.2cm}
	
	\begin{flushleft}
	Die 3 Bedeutungen im Detail: 
	\end{flushleft}
	
	\vspace{0.2cm}
	
	\begin{tabular}{lcl}
	\hline
	Symbol & Wert & Bedeutung  \\  \hline
									& 0 & weder Ziehen noch Schlagen \\
	\textcolor{black}{$\circ$}		& 1 & Ziehen \\ 
	\textcolor{black}{$\bullet$} 	& 2 & Ziehen und Schlagen  \\ 
	\end{tabular}
	 
	\vspace{0.3cm}
    
    \begin{flushleft}
    
    Beide Seiten des Spielsteins können Beschriftet sein, die normale obere Seite muss 
    markiert sein (Kreis unten rechts auf dem Stein oder Kreis rechts auf der unteren Kante).
	Die untere Spielseite des Steins stellt den beförderten Stein da, die untere Seite darf 
	von der Wertigkeit nicht geringer ausfallen als die obere Seite.
	%TODO (Maximale Punkt Differenz zwischen den Seiten?)
	%TODO Spielsteinwert = Topseite Wert + Differenz Rückseite??	
	Das Verhalten des Spielsteins wird durch die momentan oben liegende Spielseite bestimmt.
	
	
	Wieviel ein Spielstein Wert ist ergibt sich aus der Summe der generellen Wertigkeit + Wertigkeiten der einzelnen Richtungen für beide Spielstein Seiten.
	%TODO Zum Beispiel: (TODO)
	%TODO Bild Beispielstein zum Berechnen von Wert Vorder, Rückseite
	\end{flushleft}
	
	%TODO Sonderformen: Königstein
	%TODO Größe des Spielsteins
	
\subsection{Spielzüge}

	Für einzelne Spielzüge werden \textbf{Aktionspunkte }kurz \textbf{AP} benötigt, diese werden durch Würfeln bestimmt oder werden vorgegeben.
	Ein Spielzug besteht also aus dem ermitteln der Aktionspunkte und \textbf{4 Phasen} die wie folgt definiert sind.
	\begin{enumerate}
	\item Opfern \\
		  Eine Spielfigur opfern um deren Wert als AP hinzuzugewinnen.\\
		  Die Spielfigur wandert in den Besitz des Gegners (Oder wird aus dem Spiel entfernt)
	\item Bewegen \& Schlagen \\
		  Eine Spielfigur bewegen und/oder gegnerische Spielfiguren schlagen. Dabei darf nur
		  ein Spielstein in eine Richtung gezogen werden.
	\item Befördern \\
		  Eigene Spielfiguren befördern (Spielstein umdrehen)
	\item Einsetzen \\
		  Geschlagene Gegenerische Figuren werden gefangen genommen und können später
		  wieder eingesetzt werden.
	\end{enumerate}

	\begin{flushleft}
	Die Reihenfolge muss eingehalten werden, nach einer Beförderung oder dem Einsetzen darf nicht noch ein 
	Spielstein gezogen oder geschlagen werden.
	Die Kosten für die einzelne Aktionen belaufen sich wie folgt:
	\end{flushleft}

	\begin{tabular}{lp{10cm}}
	\hline
	Aktion & 
		Kosten \\ \hline
	Opfern & 
		+ n AP der Wertigkeit der Figur\newline pro Figur höchstens 3 (insgesamt höchstens 6 AP pro Zug) \\
	Bewegen & 
		- 1 AP \\
	Spielfigur schlagen & 
		- 1 AP - Differenz der generellen Wertigkeit, Ersetzt Bewegen. \\
	Befördern & 
		Differenz der generellen Wertigkeit der 2 Seiten, min. - 1 AP \\
	Einsetzen & 
		Generelle Wertigkeit des Spielsteins der eingesetzt werden soll \\ \hline
	\end{tabular} 
	
	\subsubsection*{Beispiele}
	
	\begin{tabular}{llc}
	Angreifer & Verteidiger & Kosten AP \\ \hline 
	$\blacksquare$ & $\blacksquare$ & 2 \\
	$\blacksquare$ & $\blacksquare\blacksquare$ & 3 \\
	$\blacksquare$ & $\blacksquare\blacksquare\blacksquare$ & 4 \\
	
	$\blacksquare\blacksquare$ & $\blacksquare$ & 1 \\ 
	$\blacksquare\blacksquare$ & $\blacksquare\blacksquare$ & 2 \\ 
	$\blacksquare\blacksquare$ & $\blacksquare\blacksquare\blacksquare$ & 3 \\ 
	
	$\blacksquare\blacksquare\blacksquare$ & $\blacksquare$ & 1 \\ 
	$\blacksquare\blacksquare\blacksquare$ & $\blacksquare\blacksquare$ & 1 \\ 
	$\blacksquare\blacksquare\blacksquare$ & $\blacksquare\blacksquare\blacksquare$ & 2 \\ 
	\end{tabular} 
	
	% 1 - 1 : 2
	% 2 - 1 : 1
	% 3 - 1 : 1
	% 1 - 2 : 3
	% 1 - 3 : 4
	% 2 - 3 : 2
	% 3 - 3 : 2
	
	\begin{figure}[h]
	\includegraphics[scale=0.8]{pdf/board_situation1.pdf}
	\end{figure}
	

	
	%TODO Beispiele (TODO)
	%TODO (Bewegen und Schlagen)
	%TODO Bild Spielsituation: Bewegen zu Feld X, 1 Spielstein
	%TODO Bild Spielsituation: 2 Spielsteine, schlagen
	%TODO Bild Spielsituation befördern
	%TODO Bild Spielsituation Sideboard + Board -> Einsetzen

\subsection{Das Spielbrett}
	Gespielt werden kann generell auf jedem kariertem Brett, empfohlene Standardgrößen sind
	8x8 (Schach), 9x9 (Shogi) und 10x10.
	

%%%%%%%%%%%%%%%%%%%%%%%%%%%%%%%%%%%%%%%%%%%%%%%%%%%%%%%%%%%%
%% Spielen
%%%%%%%%%%%%%%%%%%%%%%%%%%%%%%%%%%%%%%%%%%%%%%%%%%%%%%%%%%%%
\section{Das Spiel}

	Ein Spiel ist in folgende Spielphasen unterteilt.

	\begin{enumerate}
	\item Spielstart \\
	 	  Jeder Spieler wählt abwechselnd seine Spielsteine bis die passende Armee Größe erreicht ist.
	\item Aufstellung \\
	  	  Abwechselnd positioniert jeder Spieler einen Spielstein auf dem Spielbrett
	\item Spiel \\
	  	  Nach dem die Aufstellung komplett ist beginnt das eigentliche Spiel bis ein Spieler gewonnen hat.
	  	  Ein Spielzug ist wiederum unterteilt 4 Phasen.
	\end{enumerate}


\subsection{Spielstart}
	Vor Beginn jedes Spiels wird entschieden welche Spielart gespielt wird und wie groß die eingesetzte Armee sein darf,
	dies bestimmt man durch eine Vorgabe an Punkten die den Wertigkeiten der Spielsteine entsprechen.
	Jeder Spiel darf abwechselnd einen Spielstein wählen, dabei wird die Wertigkeit jedes Spielsteins addiert.
	Sobald die festgelegte Punktzahl erreicht ist darf ein Spieler keine weiteren Spielsteine mehr wählen.
	Dies wird so lange wiederholt bis beide Spieler ihre Armee als komplett ansehen.

	Zudem müssen Gewinnbedingungen, zugelassene Spielsteine und eventuelle Handicaps abgeklärt werden. 

	Folgende Elemente müssen geklärt werden:
	\begin{itemize}
	\item Spielbrett 
	\item Armeegröße 
	\item Handicaps 
	\item Zugelassene Spielsteine 
	\item Bestimmen der AP
	\item Gewinnbedingungen
	\end{itemize}
	
	Empfehlungen Armeegröße für Spielbretter:
	
	\begin{tabular}{ll}
	\hline
	Brettgröße & Empfohlene Armeepunktzahl \\ \hline
	10x10 & 200 \\
	9x9 & 150 \\
	8x8 & 100 \\ 
	\end{tabular} 
	
	\subsubsection{Spielvarianten}
	\begin{tabular}{cc}
    \hline 
    Spielvariante &  \\ 
    \hline 
    Standard & Spielbrett 9x9, Armeegröße 150
    \\ 
	\hline 
	\end{tabular} 
	

\subsection{Aufstellung}

	Abwechselnd werden Spielsteine gesetzt, dabei dürfen nur die eigenen
	hintersten 2-3 Reihen verwendet werden. Es müssen nicht alle Spielsteine
	auf dem Feld gesetzt werden, jedem Spieler ist es überlassen wieviele Spielsteine
	er aus seiner Auswahl im Sideboard lässt. Diese können später durch das bezahlen
	von Aktionspunkten im Spiel eingesetzt werden.
	
	\begin{tabular}{ll}
	\hline
	Brettgröße & Reihen \\ \hline
	10x10, 9x9 & 3 \\
	8x8        & 2 \\ 
	\end{tabular} 
	
	%TODO Bild Einsetzfläche von Spielsteinen
	%TODO Bild Beispiel fertige Aufstellung
	
\subsection{Das Spiel}

	Jeder Spiel vollzieht abwechselnd einen Spielzug, sobald alle Spieler am Zug waren
	ist eine Spielrunde zu Ende. Ein Zug besteht also aus folgenden Phasen:
	
	% ⚀ ⚁ ⚂ ⚃ ⚄ ⚅	
	% ☖ ☗
	
	\begin{enumerate}
	\item Bestimmen der Aktionspunkte (AP) \\
		  Die Aktionspunkte die ein Spieler für seinen Zug zur Verfügung hat können auf mehrere Arten bestimmt werden.
		  Es kann jeder Spieler für sich einen W6 werfen, es kann ein W6 geworfen der die AP für die Spielrunde für alle Spieler bestimmt.
		  Es kann alternativ auch eine feste Anzahl an AP für das gesamte Spiel festgelegt werden.
		  
	\item Bewegung mit einem Spielstein \\
		  Pro Spielzug darf ein Spielstein bewegt werden, die Möglichkeiten sind allerdings durch die Aktionspunkte begrenzt.
		   
	\item Befördern von Spielsteinen \\
		  Zum befördern wird ein Spielstein herum gedreht und fort an mit der unteren Seite weitergespielt.
		  Die Beförderung ist nicht umkehrbar, außer der Spielstein wird geschlagen und neu eingesetzt.
		  
	\item Einsetzen von Spielsteinen \\
		  Eingesetzt wird ein Spielstein immer mit der oberen Seite.
	\end{enumerate}
	
	
	
	\textbf{DRAFT: Combo-Spiel}, um mehr Dynamik im Spiel zu haben, 
	zum Beispiel ein Einsetzen$\rightarrow$Opfer$\rightarrow$Zug.
	Somit kann man sich unter Einsatz von Figuren zusätzliche Züge erkaufen.
	
	


\subsection{Gewinnbedingungen}

	\textit{Diese Sektion ist noch ein reiner DRAFT} \\
	Bisher keine Empfehlungen, reine Ideensammlung.
	
	\begin{itemize}
	\item Siegpunkte \\
		  Für jeden geschlagenen Spielstein werden die Punkte aufgeschrieben und addiert.

			% Idee Ansage vor dem (Würfel)Zug -> Verdoppeltn der Ergebnisse 
	
	\item Sobald eine gewisse Armeegröße (nach Punkten) gefangen genommen worden ist \\
		  Andere Variante der Siegpunkte, die Siegpunkte sind die gefangenen Figuren
	
	\item Schlagen eines Königsteins
	
	\item 'Heimbringen' von Spielsteinen \\
	       Gewisse Anzahl von Punkten über das Spielfeld bringen
	
	\item Schlagen aller gegnerischen Spielsteine \\
	      (Dauert zu lange?) Kann ewig dauern	
	\end{itemize}
	
	%anpassen der spielstärke mit steinen?
	%gewinnender spieler wird ab einem gewissen zeitpunkt
	%stärker um das spiel beenden zu können
	%combo play?
	%opfern von figuren nach dem initial zug möglich?
		%einsetzen -> opfern -> ziehen?


%%%%%%%%%%%%%%%%%%%%%%%%%%%%%%%%%%%%%%%%%%%%%%%%%%%%%%%%%%%%
%% Spielsteinliste
%%%%%%%%%%%%%%%%%%%%%%%%%%%%%%%%%%%%%%%%%%%%%%%%%%%%%%%%%%%%
\section{Spielstein-Liste}
	
	Durch die freie Konfiguration der Spielsteine sind sehr viele Kombinationen möglich, 
	aus diesem Grund gibt es hier eine vordefinierte Liste an Spielsteinen. Diese Spielsteine
	müssen akzeptiert werden, andere Konfigurationen von Spielsteinen erfordern das Einverständnis
	des Gegenspielers.
	
	\begin{tabular}{llll}
	\hline
	Vorderseite & Rückseite & Kürzel & Beschreibung \\ \hline
	• & • & • \\ 
	\hline
	\end{tabular} 

%%%%%%%%%%%%%%%%%%%%%%%%%%%%%%%%%%%%%%%%%%%%%%%%%%%%%%%%%%%%
%% Notation
%%%%%%%%%%%%%%%%%%%%%%%%%%%%%%%%%%%%%%%%%%%%%%%%%%%%%%%%%%%%
\section{Notation}

\textit{Diese Sektion ist noch ein reiner DRAFT}


Pick-Phase:

Pick N-NE-E-SE-S-SW-W-NW-V
PI 110000014-220000025
PI - 

Place-Phase
 
PL 110000015 8g
PL -

Game Phase:

S* 8g-7g
W 
PL

\begin{tabular}{|l|c||}
\hline Zug & Werte \\ \hline
• & • \\  
\end{tabular} 


\section{Etikette}
	Verhaltensregeln während eines Spiels.
	
	

%%%%%%%%%%%%%%%%%%%%%%%%%%%%%%%%%%%%%%%%%%%%%%%%%%%%%%%%%%%%
%% Glossar
%%%%%%%%%%%%%%%%%%%%%%%%%%%%%%%%%%%%%%%%%%%%%%%%%%%%%%%%%%%%
\section{Glossar}

%%%%%%%%%%%%%%%%%%%%%%%%%%%%%%%%%%%%%%%%%%%%%%%%%%%%%%%%%%%%
%% Lizenz
%%%%%%%%%%%%%%%%%%%%%%%%%%%%%%%%%%%%%%%%%%%%%%%%%%%%%%%%%%%%
\section{Lizenz}
Creative Commons Lizenzvertrag
Diese(s) Werk bzw. Inhalt von Okard steht unter einer Creative Commons Namensnennung-NichtKommerziell-KeineBearbeitung 3.0 Unported Lizenz.
\href{http://creativecommons.org/licenses/by-nc-nd/3.0/}{CC BY-NC-ND 3.0}



\end{document}