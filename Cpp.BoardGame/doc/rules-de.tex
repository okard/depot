\documentclass{article}
\usepackage[a4paper]{geometry}
\usepackage[utf8]{inputenc}
\usepackage[T1]{fontenc}
\usepackage[ngerman]{babel}
\usepackage{amsmath}
\usepackage{mathptmx} %times font
\usepackage{graphicx}
 
\title{Spielregeln}
\author{okard}
\date{\today}
\begin{document}

Dieses Werk bzw. Inhalt steht unter einer Creative Commons Namensnennung-Nicht-kommerziell-Weitergabe unter gleichen Bedingungen 3.0 Unported Lizenz.
http://creativecommons.org/licenses/by-nc-sa/3.0/

\section{Spielfiguren}

\begin{figure}[ht]
	\centering
	\includegraphics{img/01_BasicToken_svg.pdf}
	\caption{Spielfigur}
	\label{Figure1}
\end{figure}


\section{Spielstart}
Einigen auf Brettgröße, in der Regel 8x8 oder 10x10.
Einigen auf Anzahl der Armeepunkte

\section{Pick-Phase}
Bevor das Spielstart wählt jeder Spieler abwechselnd aus seinem Repertoire eine Spielfigur aus, die Armeekosten der Spielfiguren werden summiert, sobald ein Spieler die vorher festgelegte Armeepunkte erreicht hat (sie darf nicht überschritten werden)

\section{Set-Phase}
Die Figuren dürfen abwechselt jeweils in den ersten 2 Reihen verteilt werden. Es müssen nicht alle Figuren gesetzt werden, der Spieler kann entscheiden wie viele er setzen möchte. Maximal dürfen 2 x Spielfeldbreite an Figuren gesetzt werden, also für 8x8 maximal 16.

\section{Spielrunden}

Eine Spielrunde besteht aus 4 Phasen

\subsection{Würfeln}
Zu Beginn jeder Runde wird mit einem W6 (Würfel mit 6 Seiten) gewürfelt, 
die Augenzahl entspricht der für diese Runde verfügbaren Aktionspunkten 
	
\subsection{Bewegen \& Schlagen}
Jeder Zug kostet für jede Figur eine spezifische Anzahl an Aktionspunkten

Jede Bewegung kostet 1 Aktionspunkt pro Schritt, jede Figur darf pro Zug höchstens 1 mal in 1 Richtung bewegt werden.

Für das Schlagen gelten folgende Regeln
Das Schlagen in diese Richtung muss erlaubt sein, falls der Basiswert kleiner gleich ist kann mit 1 Aktionspunkt für die Bewegung geschlagen werden. Ansonsten muss die Differenz durch Aktionspunkte ausgeglichen werden. Wenn eine Figur geschlagen wird, darf man sie gefangen nehmen und kann später wieder eingesetzt werden auf der eigenen Seite.
	
\subsection{Einsetzen}
Das einsetzen einer Figur kostet die Anzahl an Aktionspunkten die dem Basiswert der Figur entsprechen. Eingesetzt werden können die eigenen Figuren die bisher nicht am Spielgeschehen teilgenommen haben oder gefangene gegnerische Figuren.
	
\subsection{Up/Downgraden}
Das Up oder Downgraden kostet die Anzahl an Aktionspunkten die dem Basiswert der Figur entsprechen. Das up/Downgraden der Figuren findet durch umdrehen statt.

\section{Spielende}
Aufgeben: Ein Spieler darf jederzeit aufgeben.


\begin{flushleft}
Spielvarianten:

\textbf{Siegpunkte}: Hier wird bei jedem Schlagen der komplette Wert der Spielfigur aufgeschrieben, wer als erstes die Anzahl an Punkten erreicht hat die der Armeestärke entspricht hat gewonnen.

\textbf{Zerstörung}: Alle Spielsteine des Gegners sind geschlagen worden.

\textbf{Der Weg}: Einen Spielstein auf die andere Seite des Spielbretts bringen
\end{flushleft}


\section{Umgangsregeln}
Die Seiten eigener Steine dürfen im eigenen Zug jederzeit eingesehen das darf auch verdeckt geschehen, nach dem Anheben ist die Figur wieder auf ihren ursprünglichen Platz zu stellen. Man darf den Gegner bitten auch einen seiner Steine einsehen zu dürfen, dieser kann die bitte gewehren oder ablehnen.

\section{Maße}
Maße für Spielsteine und Brett

	

\end{document}
